%%%%%%%%%%%%%%%%%%%%%%%%%%%%%%%%%%%%%%%%%
% Beamer Presentation
% LaTeX Template
% Version 1.0 (10/11/12)
%
% This template has been downloaded from:
% http://www.LaTeXTemplates.com
%
% License:
% CC BY-NC-SA 3.0 (http://creativecommons.org/licenses/by-nc-sa/3.0/)
%
%%%%%%%%%%%%%%%%%%%%%%%%%%%%%%%%%%%%%%%%%

% References:
%  - https://en.wikibooks.org/wiki/LaTeX/Presentations (wikibooks beamer manual)
%  - http://texdoc.net/texmf-dist/doc/latex/beamer/doc/beameruserguide.pdf (beamer manual -- official guide)
%  - https://www.latex-project.org/publications/fmittelbach-1a-2017.pdf (sample beamer presentation)

\documentclass[12pt,compress,final]{beamer}
% \documentclass[12pt,compress,final,handout]{beamer}

\mode<presentation> {

%----------------------------------------------------------------------------------------
%	PACKAGES AND THEMES
%----------------------------------------------------------------------------------------

% The Beamer class comes with a number of default lide themes
% which change the colors and layouts of slides. Below this is a list
% of all the themes, uncomment each in turn to see what they look like.

% reference and preview: http://deic.uab.es/~iblanes/beamer_gallery/index_by_theme_and_color.html (sample themes and colors)

%\usetheme{default}
%\usetheme{AnnArbor}
%\usetheme{Antibes}
%\usetheme{Bergen}
%\usetheme{Berkeley}
%\usetheme{Berlin}
%\usetheme{Boadilla}
%\usetheme{CambridgeUS}
%\usetheme{Copenhagen}
%\usetheme{Darmstadt}
%\usetheme{Dresden}
\usetheme{Frankfurt}
%\usetheme{Goettingen}
%\usetheme{Hannover}
%\usetheme{Ilmenau}
%\usetheme{JuanLesPins}
%\usetheme{Luebeck}
%\usetheme{Madrid}
%\usetheme{Malmoe}
%\usetheme{Marburg}
%\usetheme{Montpellier}
%\usetheme{PaloAlto}
%\usetheme{Pittsburgh}
%\usetheme{Rochester}
%\usetheme{Singapore}
%\usetheme{Szeged}
%\usetheme{Warsaw}

% As well as themes, the Beamer class has a number of color themes
% for any slide theme. Uncomment each of these in turn to see how it
% changes the colors of your current slide theme.

%\usecolortheme{albatross}
\usecolortheme{beaver}
%\usecolortheme{beetle}
%\usecolortheme{crane}
%\usecolortheme{dolphin}
%\usecolortheme{dove}
%\usecolortheme{fly}
%\usecolortheme{lily}
%\usecolortheme{orchid}
%\usecolortheme{rose}
%\usecolortheme{seagull}
%\usecolortheme{seahorse}
%\usecolortheme{whale}
%\usecolortheme{wolverine}

%\setbeamertemplate{footline} % To remove the footer line in all slides uncomment this line
\setbeamertemplate{footline}[page number] % To replace the footer line in all slides with a simple slide count uncomment this line

%\setbeamertemplate{navigation symbols}{} % To remove the navigation symbols from the bottom of all slides uncomment this line
%\beamertemplatenavigationsymbolsempty

% grey out items after pause
\setbeamercovered{transparent,highly dynamic}

% reduce spacing between figure/table and captions
% https://tex.stackexchange.com/questions/226127/how-to-remove-spacing-between-figure-and-caption-in-the-beamer-class
% \setlength\abovecaptionskip{-5pt}

\setbeamertemplate{itemize items}[default]
\setbeamertemplate{enumerate items}[default]

\setbeamertemplate{navigation symbols}[only frame symbol]
}

\usepackage{iftex}
\ifPDFTeX
  % use UTF-8 in pdftex (not needed in LuaTeX and XeTeX)
  \usepackage[utf8]{inputenc}
\fi

% document language (the latest option is set as the current active language)
\usepackage[english,portuguese]{babel}

% math and symbols
\usepackage{amsmath,amssymb}

% next-generation computer modern fonts
\usepackage{lmodern}

% Allows including images
\usepackage{graphicx}

% required for better tables
\usepackage{longtable}
\usepackage{multirow}
\usepackage{makecell}
\usepackage{colortbl} % highlight rows: \rowcolor{}

% curly brackets besides tables (\ldelim, \rdelim)
\usepackage{bigdelim}

\usepackage[font=small]{caption}

% Allows the use of \toprule, \midrule and \bottomrule in tables
\usepackage{booktabs}

% text highlighting, underlining, spacing
\usepackage{soul}

% todo notes
\usepackage[obeyFinal,portuguese,textsize=footnotesize]{todonotes}

% custom todo notes macros
\newcommand{\todox}[2][]{\sethlcolor{yellow}\texthl{#1}\todo[author=\textbf{TODO},inline,color=yellow]{#2}}
\newcommand{\perrotta}[2][]{\sethlcolor{pink}\texthl{#1}\todo[author=\textbf{Thiago Perrotta},inline,color=pink]{#2}}
\newcommand{\silva}[2][]{\sethlcolor{lightgray}\texthl{#1}\todo[author=\textbf{Vitor Silva},inline,color=lightgray]{#2}}

% \listoftodos{} is incompatible with beamer
% source: https://tex.stackexchange.com/questions/313426/list-of-todos-in-beamer

% speaker notes
% reference: https://gist.github.com/andrejbauer/ac361549ac2186be0cdb
\usepackage{pgfpages}
\setbeamertemplate{note page}{\pagecolor{yellow!5}\insertnote}
%\setbeamertemplate{note page}[plain]

% These slides also contain speaker notes. You can print just the slides, just the notes, or both, depending on the setting below. Comment out the one you want.
%\setbeameroption{hide notes} % Only slides
%\setbeameroption{show notes}
%\setbeameroption{show only notes} % Only notes
%\setbeameroption{show notes on second screen=right} % Both

\usepackage{listings}
\renewcommand{\lstlistingname}{Código}
\lstset{basicstyle=\footnotesize\sf}

\lstset{%
  aboveskip=15pt,
  belowskip=15pt,
  breakatwhitespace=false,     % sets if automatic breaks should only happen at whitespace
  breaklines=true,             % sets automatic line breaking
  postbreak=\mbox{\textcolor{red}{$\hookrightarrow$}\space},
  captionpos=t,                % sets the caption-position
  columns=fullflexible,
  commentstyle=\color{gray},   % sets the comments color
  escapechar=\%,               % write LaTeX inside listings
  frame=lines,                 % adds a frame
  numbers=left,                % where to put the line-numbers
  numberstyle=\footnotesize\color{gray}, % the size of the fonts that are used for the line-numbers (alt: \tiny)
  showspaces=false,            % show spaces adding particular underscores
  showstringspaces=false,      % underline spaces within strings
  showtabs=false,              % show tabs within strings adding particular underscores
  stepnumber=1,                % the step between two line-numbers. If it's 1 each line
  stringstyle=\color{gray},
  tabsize=2,                   % sets default tabsize to 4 spaces
  title=\lstname,              % show the filename of files included with \lstinputlisting;
}

\lstdefinelanguage{sqlresults}{%
  aboveskip=5pt,
  belowskip=5pt,
  basicstyle=\footnotesize\ttfamily,
  columns=fixed,
  numbers=none,
  frame=none,
}

% hyperlinks
\hypersetup{%
  portuguese,
  pdfencoding=auto,
  breaklinks=true,
  bookmarksopen=true,
  colorlinks=true, % false: boxed links; true: colored links
  linkcolor=, % https://tex.stackexchange.com/questions/13423/how-to-change-the-color-of-href-links-for-real
  pdfstartview={FitH}, % fits the width of the page to the window
  pdfauthor={Thiago Barroso Perrotta},
  pdfcreator={Thiago Barroso Perrotta},
  pdfproducer={Thiago Barroso Perrotta},
  pdftitle={Análise do Rastro de Proveniência em Simulações Computacionais em Larga Escala},
  pdfsubject={Análise do Rastro de Proveniência em Simulações Computacionais em Larga Escala},
  pdfkeywords={Workflows Científicos,Análise de Dados Científicos,Gerência de Fluxos de Dados,Dados de Proveniência,Bancos de Dados},
}

%----------------------------------------------------------------------------------------
%	TITLE PAGE
%----------------------------------------------------------------------------------------
\title[Análise do Rastro de Proveniência]{Análise do Rastro de Proveniência em Simulações Computacionais em Larga Escala} % The short title [] appears at the bottom of every slide, the full title is only on the title page

\author[Thiago B. Perrotta]{Thiago Barroso Perrotta} % Your name
\institute[UFRJ] % Your institution as it will appear on the bottom of every slide, may be shorthand to save space
{
Engenharia de Computação e Informação \\ \smallskip
Escola Politécnica --- Universidade Federal do Rio de Janeiro \\ % Your institution for the title page
\medskip
\href{mailto:perrotta.thiago@poli.ufrj.br}{\nolinkurl{perrotta.thiago@poli.ufrj.br}} % Your email address
}
\date{14 de Setembro de 2017} % Date, can be changed to a custom date

\pgfdeclareimage[height=1.5cm]{university-logo}{escola-politecnica-ufrj}
\logo{\pgfuseimage{university-logo}}

% Environment to remove the headline from a frame
% https://tex.stackexchange.com/questions/44983/beamer-removing-headline-and-its-space-on-a-single-frame-for-plan-but-keepin
\makeatletter
    \newenvironment{withoutheadline}{
        \setbeamertemplate{headline}[default]
        \def\beamer@entrycode{\vspace*{-\headheight}}
    }{}
\makeatother

% https://tex.stackexchange.com/questions/53781/how-can-i-include-the-logo-in-some-slides-and-remove-in-others-using-beamer
\newcommand{\nologo}{\setbeamertemplate{logo}{}} % command to set the logo to nothing

% Prepend TOC to every section
% \AtBeginSection[] {
%    \begin{withoutheadline}
%    \setbeamertemplate{footline}{}
%    \begin{frame}<beamer>{Agenda}
%    \tableofcontents[currentsection]  
%    \end{frame}
%    \end{withoutheadline}
% }

% \includeonlyframes{
% current,
% label1,
% label2,
% }

\begin{document}

% Print the title page as the first slide
% https://tex.stackexchange.com/questions/33767/remove-section-header-from-a-beamer-theme-singapore
{
\setbeamertemplate{headline}{}
\setbeamertemplate{footline}{}
\begin{frame}
  \titlepage{}
\end{frame}
% \frame{\titlepage}
\addtocounter{page}{-1}
}

%------------------------------------------------

% Print the table of contents
\begin{withoutheadline}
\setbeamertemplate{footline}{}
\begin{frame}{Agenda}
  \tableofcontents{} % Throughout your presentation, if you choose to use \section{} and \subsection{} commands, these will automatically be printed on this slide as an overview of your presentation
\end{frame}
\end{withoutheadline}

%------------------------------------------------

\section{Introdução}
\subsection*{Introdução}

{\nologo
\begin{frame}{Introdução}{Análise do \alert{Rastro de Proveniência} em \alert{Simulações Computacionais} em Larga Escala}

    \begin{alertblock}{Simulação computacional}
    \textit{``Método de abstração e sistematização do processo experimental ou parte dele''}
    \smallskip
    \begin{itemize}
        \item experimentos científicos em ambientes computacionais
        \item fluxo de dados
    \end{itemize}
    \end{alertblock}
    
    \begin{columns}
    \column{.7\textwidth}
    \begin{block}{Rastro de proveniência}
    \begin{itemize}
        \item Monitoramento da simulação durante a sua execução
        \item Reconstrução do fluxo\ldots{}
        \begin{itemize}
            \item \ldots{}de arquivos % nível físico
            \item \ldots{}de dados % nível lógico
        \end{itemize}
    \end{itemize}
    \end{block}
    \column{.3\textwidth}
    \begin{figure}
    \includegraphics[width=.9\textwidth]{img/experiments-dataflow.pdf}
    \end{figure}
    \end{columns}
    
\end{frame}
}

%------------------------------------------------

\subsection*{Motivação}

\begin{frame}{Motivação}

\begin{itemize}
    \item Simulações computacionais levam um \alert{longo tempo} para serem executadas % mesmo em PAD
    \item Torna-se necessário extrair resultados mesmo \emph{antes} do término das mesmas
    \item Avaliar grandezas tais como resíduos ou estimativas de erro
    \item Soluções existentes são difíceis de serem configuradas pelos usuários
\end{itemize}

\end{frame}

%------------------------------------------------

\section{Referencial Teórico}
\subsection*{Referencial Teórico}

\begin{frame}[t]{Fluxo de dados}

\centerline{$D = (T, S, \Phi)$}

$\left\{
\begin{tabular}{p{\textwidth}}
\begin{itemize}
    \item \( T = \{dt_1, dt_2, \ldots, dt_{\alpha}\} \) : conjuntos de \alert{transformações}
    \item \( S = \{ds_1, ds_2, \ldots, ds_{\beta}\} \) : \alert{conjuntos de dados}
    \item \( \Phi = \{\phi_1, \phi_2, \ldots, \phi_{\gamma}\} \) : \alert{dependências} de dados
\end{itemize}
\end{tabular}
\right.$

\vfill

\begin{itemize}
\item \alert{Grafo} direcionado acíclico:
\begin{itemize}
\item nós $\leftrightarrow$ transformações de dados
\item arestas $\leftrightarrow$ conjuntos de dados
\end{itemize}
\end{itemize}

\begin{figure}
\includegraphics[width=.35\textwidth]{img/graph.pdf}
\end{figure}

\note[item]{Pode representar uma simulação computacional}
\end{frame}

%------------------------------------------------

% \begin{frame}[t]{Transformação de dados}
% 
% \begin{itemize}
% \item \alert{consome} conjuntos de dados (entrada)
% \item \alert{produz} conjuntos de dados (saída)
% \item corresponde a um ou mais \alert{programas} de simulação
% \end{itemize}
% 
% \vfill
% 
% \begin{figure}
% \includegraphics[width=.5\textwidth]{img/example-data-transformation.pdf}
% % \caption{Transformação de dados com dois conjuntos de dados de entrada e dois de saída}
% \end{figure}
% 
% \end{frame}

%------------------------------------------------

{\nologo
\begin{frame}[t,squeeze]{Conjunto de dados}

\centerline{$ds = (A, C)$}

$\left\{
\begin{tabular}{p{\textwidth}}
\begin{itemize}
    \item \( A = \{a_1, a_2, \ldots, a_{\delta} \} \) : \alert{atributos} de dados
    \item \( C = \{dc_1, dc_2, \ldots, dc_{\zeta} \} \) : \alert{coleções de dados}
\end{itemize}
\end{tabular}
\right.$

\vfill

\begin{figure}
\includegraphics[width=.45\textwidth]{img/example-data-transformation.pdf}
% \caption{Transformação de dados com dois conjuntos de dados de entrada e dois de saída}
\end{figure}

\vfill

\begin{block}{Atributos de dados}
\begin{itemize}
\item \( \textup{a} = (\textup{nome},\textup{tipo}) \)
\item $\textup{\alert{tipo}} \in \{\textup{\small inteiro, ponto flutuante, texto, arquivo, booleano}\}$
\end{itemize}
\end{block}

\end{frame}
}

%------------------------------------------------

% {\nologo
% \begin{frame}{Atributo de dados}
% 
% \centerline{\( A = \{a_1, a_2, \ldots, a_{\delta} \} \)}
% 
% $\left\{
% \begin{tabular}{p{\textwidth}}
% \begin{itemize}
% \item $\textup{da} = (\textup{nome},\textup{tipo})$
% \item $\textup{\alert{tipo}} \in \{\textup{\small inteiro, ponto flutuante, texto, arquivo, booleano}\}$
% \end{itemize}
% \end{tabular}
% \right.$
% 
% \vfill
% 
% \begin{exampleblock}{Exemplo}
% \begin{table}[htb]
%     \centering
%     \begin{tabular}{|c|c|}
%         \hline
%         \textbf{Nome do atributo} & \textbf{Tipo do atributo} \\
%         \hline
%         Nome             & texto           \\
%         \hline
%         Idade            & inteiro         \\
%         \hline
%         Altura \( (m) \) & ponto flutuante \\
%         \hline
%     \end{tabular}
% \end{table}
% \end{exampleblock}
% 
% \end{frame}
% }

%------------------------------------------------

\begin{frame}{Coleção e elemento de dados}

\begin{block}{Coleção de dados}
\begin{itemize}
\item $dc = \{ de_1, de_2, \ldots, de_{\eta} \}$ : \alert{elementos} de dados
\end{itemize}
\end{block}

\begin{block}{Elemento de dados}
\begin{itemize}
\item $de = \{ v_1, v_2, \ldots, v_{\theta} \} \mid \#(ds.A) = \#(de)$ : valores
\end{itemize}
\end{block}

\vfill

\begin{table}[htb]
    \centering
    \begin{tabular}{c|c|c|cc}
        & \textbf{Nome} & \textbf{Idade} & \textbf{Altura \((m)\)} \\
        \( de_{1} \) & Alice   & 22 & \( 1,77 \) & \rdelim\}{2}{.1mm}[\( dc_{1} \)] \\
        \( de_{2} \) & Bob     & 25 & \( 1,79 \) \\
        \( de_{3} \) & Charlie & 42 & \( 1,85 \) & \rdelim\}{1}{.1mm}[\( dc_{2} \)] \\
    \end{tabular}
    \caption{\( dc_{1} = \{de_{1}, de_{2}\} \) e \( dc_{2} = \{ de_{3} \} \)}
\end{table}

\end{frame}

%------------------------------------------------

{\nologo
\begin{frame}{Dados de proveniência}{Especificação}

\begin{alertblock}{Proveniência $\approx$ \textsc{causalidade}}
\textit{``Lugar de onde alguém ou alguma coisa provém; origem; fonte.''} \\
\hspace*{\fill}--- \textsc{Dicionário Aulete}
\end{alertblock}

\vfill

\begin{exampleblock}{Tipos de dados de proveniência}
\centerline{Diferentes níveis de abstração:}
\begin{description}[retrospectiva]
\item[prospectiva] \alert{processo computacional}, passo-a-passo
\item[retrospectiva] contexto de \alert{execução}
\end{description}
\end{exampleblock}

% \begin{block}{Captura de dados de proveniência}
% \begin{itemize}
% \item \textit{online} $\rightarrow$ flexível
% \item \textit{offline} $\rightarrow$ desempenho
% \end{itemize}
% \end{block}

\note[item]{\{ambiente, tempo\} de execução}
\end{frame}
}

%------------------------------------------------

% \begin{frame}[t]{Dados de proveniência}{Exemplo}
% 
% \centerline{\footnotesize \textbf{Proveniência Prospectiva}}
% 
% \begin{figure}
% \includegraphics[width=\textwidth]{img/word-count-prospective.pdf}
% \end{figure}
% 
% \begin{table}[htb]
%     \centering
%     \footnotesize
%     \begin{tabular}{r|l}
%         \multicolumn{2}{c}{\textbf{Proveniência Retrospectiva}} \\
%         \midrule
%         \textit{Timestamp} de início & 2017-06-25 03:11:05.231                         \\
%         \textit{Timestamp} de fim    & 2017-06-25 03:11:09.586                         \\
%         PID                          & 4671                                            \\
%         Linha de comando             & \texttt{./sort --in=freq-list.csv} \\
%         Variáveis de ambiente        & \texttt{PWD=/exec:PATH=/usr/bin} \\
%         Usuário                      & tperrotta                                       \\
%         \bottomrule
%     \end{tabular}
% \end{table}
% 
% \end{frame}

%------------------------------------------------

\section{Solução} % alt: Proposta, Problema

% Rastreamento de dados de proveniência através de mapeamento de atributos

% Vitor: O importante é enfatizar o rastreio do fluxo de elementos de dados em uma simulação computacional por meio da abstração de fluxo de dados (no nível lógico). Do ponto de vista de implementação, temos que valorizar as junções "automáticas" (a partir da análise do mapeamento de atributos) e a facilidade em desenvolver uma consulta na nossa abordagem.

% [OK] Marta: Apesar de a ARMFUL ser bem abrangente e eficiente em prover
% consultas envolvendo dados de domínio, de proveniência e de
% execução, ela oferece um recurso limitado para o usuário submeter
% consultas dessas três classes. Na verdade, o usuário precisa não
% apenas conhecer a linguagem SQL, mas conhecer bem os
% relacionamentos que permitem a geração dos relacionamentos
% entre múltiplos arquivos de dados.
% Esse trabalho ter por objetivo facilitar essa interação do usuário
% com a elaboração de consultas analíticas ao utilizar a ARMFUL.
% Comment [2]: Antes de falar da
% contribuição, vc poderia dizer que existem
% soluções para gerar código SQL
% automaticamente, citando-as e se possível,
% explicando porque não foram usadas.

\subsection*{O problema}

\begin{frame}[t]{O problema}
    \begin{exampleblock}{Três tipos de consulta}
    \begin{enumerate}
        \item Análise de \alert{dados científicos} proveniente de apenas um arquivo % $\mid$ Conteúdo de domínio
        \item Rastreamento do \alert{fluxo de múltiplos arquivos}
        \item Rastreamento de \alert{elementos de dados relacionados} em múltiplos arquivos
    \end{enumerate}
    \end{exampleblock}
    \vfill
    \setbeamercovered{invisible}
%     \pause
    \centerline{\textit{\large ``Como realizar esses três tipos de consulta?''}}
    \vfill
\end{frame}

%------------------------------------------------

\subsection*{A arquitetura ARMFUL}

\begin{frame}[t]{A arquitetura ARMFUL}

\begin{figure}
\includegraphics[width=\textwidth]{img/armful-architecture-simplified.pdf}
\end{figure}

\note[item]{Baseada em componentes}
\note[item]{Suporta os 3 tipos de consulta mencionados}
\note[item]{\textit{On-line} e \textit{off-line}}
\note[item]{Leitura do conteúdo, tokenização, filtragem de conteúdo, análise}
\note[item]{Correlação em um SGBD relacional: dados de domínio e dados de proveniência do fluxo de dados}
\note[item]{Gerência de conteúdo científico $\leftarrow$ arquivos de dados científicos}
\end{frame}

%------------------------------------------------

\subsection*{DfAnalyzer: uma instância da ARMFUL}

\begin{frame}{DfAnalyzer: uma instância da ARMFUL}
\begin{itemize}
\item Metadados $\leftarrow$ arquivos JSON
\begin{itemize}
\item Dependências de dados
\item Dados de proveniência
\end{itemize}
\item MonetDB
\begin{itemize}
\item SGBD relacional
\item Orientado e otimizado a colunas
\item Indexação: \textit{lazy}
\end{itemize}
\item \alert{QPP: Pré-processador de consultas}
\end{itemize}

\note[item]{MonetDB: lazy indexing / optimization}
\end{frame}

%------------------------------------------------

\subsection*{QPP: Pré-processador de consultas}

\begin{frame}[t]{QPP: Pré-processador de consultas}
% QPP, Citar o objetivo do trabalho, descrever eventuais vantagens que serão buscadas

\centerline{\large{\textsc{Proposta}}}

\begin{columns}[t]
    
\begin{column}{0.5\textwidth}

\centerline{\textbf{Problemas}}

\begin{itemize}
\item Expressar manualmente consultas em SQL
    \begin{itemize}
    \item enfadonho
    \item propenso a erros
    \end{itemize}
\item Identificar relacionamentos
    \begin{itemize}
    \item complexo
    \end{itemize}
\item É necessário conhecer o esquema do BD
\end{itemize}

\end{column}

\vrule{}

\begin{column}{0.5\textwidth}

\centerline{\textbf{Soluções}}

\begin{itemize}
    \item Consultas em SQL geradas \alert{automaticamente}
    \item Junções ``automáticas'' baseadas no \alert{rastreio de dados de proveniência}
    \item Sem necessidade de se conhecer a fundo o esquema do BD
\end{itemize}

\end{column}

\end{columns}

\end{frame}

%------------------------------------------------

\subsection*{Metodologia}

{\nologo
\begin{frame}[t]{Metodologia -- Mapeamento de atributos}

\only<1 | handout:1>{
\begin{alertblock}{\textbf{Definição}: mapeamento de atributos}
\centerline{$ds_1.a_I \leftrightarrow ds_2.a_{II}$}
\smallskip
\begin{enumerate}
\item mesmo identificador ($a_I$ e $a_{II}$)
\item mesmo tipo ($a_I$ e $a_{II}$)
\item conjuntos de dados adjacentes ($ds_1$ e $ds_2$)
\end{enumerate}
\end{alertblock}

\begin{block}{Tipos de mapeamento de atributos}
\begin{description}[híbrido]
\item[físico]  atributos relacionados à mesma instância de execução ou tarefa
\item[lógico]  atributos de domínio
\item[híbrido] \emph{ambos} os anteriores
\end{description}
\end{block}
}

% \only<2 | handout:2>{
% \vspace{-.6cm}
% 
% \begin{figure}
% \includegraphics[width=.6\textwidth]{img/example-query-dataflow.pdf}
% \end{figure}
% 
% \begin{table}[!htb]
%     \small
%     \centering
%     \begin{tabular}{c|c|c}
%     \textbf{Conjunto de dados} & \textbf{Nome do atributo} & \textbf{Tipo do atributo} \\ \hline
% \rowcolor{red!10}
% $ds_1$                        & $da_1$                        & inteiro                 \\
% $ds_1$                        & $da_2$                        & booleano                \\
% \rowcolor{blue!10}
% $ds_1$                        & $\textup{taskid\_}dt_1$      & inteiro                 \\ \hline
% \rowcolor{red!10}
% $ds_2$                        & $da_1$                        & inteiro                 \\
% $ds_2$                        & $da_2$                        & inteiro                 \\
% \rowcolor{blue!10}
% $ds_2$                        & $\textup{taskid\_}dt_1$      & inteiro                 \\
% \rowcolor{green!10}
% $ds_2$                        & $\textup{taskid\_}dt_2$      & inteiro                 \\ \hline
% \rowcolor{green!10}
% $ds_4$                        & $\textup{taskid\_}dt_2$      & inteiro                  
%     \end{tabular}
% \end{table}
% }

\end{frame}
}

%------------------------------------------------

\section{Desenvolvimento}
\subsection*{Implementação do QPP}

{\nologo
% \setbeamertemplate{footline}{}
\begin{frame}[t,squeeze]{Implementação do QPP}
\vspace{-.5cm}
\small

\begin{block}{Estrutura}
\begin{itemize}
\item Java 8
\begin{itemize}
\item Orientação a objetos
\item Consolidada na academia
\item 23 classes $\mid$ 2000 SLOC
\item 152 testes (\textit{JUnit})
\end{itemize}
\item Integração ao DfAnalyzer
\end{itemize}
\end{block}

\begin{exampleblock}{Fluxo básico de execução}
\begin{enumerate}
\item Conexão do Java com o MonetDB
\item Carregamento dos metadados do fluxo de dados
\item Instanciação do QPP
\item Execução de sucessivas \alert{consultas} (\texttt{generateSqlQuery})
\end{enumerate}
\end{exampleblock}
\end{frame}
}

%------------------------------------------------

\begin{frame}{Implementação do QPP}{Algumas técnicas adotadas}

\begin{itemize}
\item \textit{Memoization} (``\textit{caching}'')
\vfill
\item \textit{Hash Multimap}
\begin{itemize}
\item Chave-valor
\item $O(1)$ (em geral${}^{\ast}$)
\end{itemize}
\vfill
\item Busca em profundidade (DFS) no DAG do fluxo de dados
\vfill
\end{itemize}

\end{frame}

%------------------------------------------------

\subsection*{Função de geração das consultas do QPP}

\begin{frame}[fragile]{Função de geração das consultas do QPP}{Estrutura básica da cláusula SQL}

\begin{lstlisting}[language=sql,deletendkeywords={TIME},basicstyle=\large]
SELECT ds%${}_1$.A%, ds%${}_2$.B%     -- %projeções%
FROM ds%${}_1$%, ds%${}_2$%    -- %computado automaticamente%
WHERE ds%${}_1$.A %>% 10%      -- %seleções%
\end{lstlisting}

\end{frame}

%------------------------------------------------

\begin{frame}[t]{Função de geração das consultas do QPP}{\texttt{generateSqlQuery}}

\vspace{-1cm}

\begin{table}
\centering
\begin{tabular}{r|l|c}
\toprule
\textbf{Argumento} & \textbf{Tipo} & \textbf{Exemplo}               \\
\midrule
D                  & fluxo de dados              \\
\alert{dsOrigins}          & lista de conjuntos de dados & \texttt{[inputmesh]} \\
\alert{dsDestinations}     & lista de conjuntos de dados & \texttt{[outputmesh]} \\
type               & físico, lógico ou híbrido   \\
projections        & lista de strings & [``ds\textsubscript{1}.attr'']             \\
selections         & lista de strings & [``ds\textsubscript{2}.a\textsubscript{2} $>$ 10'']            \\
dsIncludes         & lista de conjuntos de dados & [ds\textsubscript{1}, ds\textsubscript{2}] \\
dsExcludes         & lista de conjuntos de dados \\
\bottomrule
\end{tabular}
\end{table}

\vfill

\begin{itemize}
\item Produto cartesiano dos possíveis caminhos entre dois conjuntos de dados
\end{itemize}

\end{frame}

%------------------------------------------------

\begin{frame}[t]{Função de geração das consultas do QPP}{Exemplos de produto cartesiano}

\vspace{-.75cm}

\begin{columns}[t]

\column{0.75\textwidth}

\begin{figure}
\includegraphics[width=0.85\textwidth]{img/example-query-dataflow-1.pdf}
\end{figure}

\column{0.25\textwidth}

\bigskip
\bigskip
ds\textsubscript{1} $\vdash$ ds\textsubscript{4}

\end{columns}

\hrule{}

\begin{columns}[t]

\column{0.75\textwidth}

\begin{figure}
\includegraphics[width=0.85\textwidth]{img/example-query-dataflow-2.pdf}
\end{figure}

\column{0.25\textwidth}

\bigskip
\bigskip
ds\textsubscript{2} $\vdash$ ds\textsubscript{5}, ds\textsubscript{6}

\end{columns}

\end{frame}

%------------------------------------------------

\section{Experimentos}

\subsection*{Simulação computacional utilizada}
\begin{frame}[t]{Simulação computacional utilizada}{Sedimentação de dinâmica de fluidos computacionais}

\vspace{-.35cm}

\only<1 | handout:1>{
\begin{alertblock}{Objetivo}
\begin{itemize}
\item Simular a turbidez e a perturbação de correntes de fluidos computacionais
\item Observar a \textbf{evolução ao longo do \alert{tempo} entre \alert{fluidos} e \alert{sedimentos}}
\end{itemize}
\end{alertblock}

% https://tex.stackexchange.com/questions/94016/how-to-reduce-space-between-image-and-its-caption
\captionsetup[figure]{skip=-5pt}
\begin{figure}
\includegraphics[width=.9\textwidth]{img/sedimentation_over_time.png}\hspace*{\fill}
\caption{Evolução da sedimentação ao longo do tempo}
\end{figure}
}

% \only<2 | handout:2>{
% \begin{block}{Como?}
% \begin{itemize}
% \item \texttt{libMesh}: refinamento de malhas e elementos finitos adaptativos
% \item equação de incompressibilidade
% \item equação de transporte
% \end{itemize}
% \end{block}
% 
% \captionsetup[figure]{skip=-5pt}
% \begin{figure}
% \includegraphics[width=.9\textwidth]{img/sedimentation.png}\hspace*{\fill}
% \caption{Ilustração representativa da dinâmica de fluidos}
% \end{figure}
% }

\note[item]{turbidez encontrada em processos geológicos}
\note[item]{libMesh: biblioteca em C++}
\note[item]{descritos por um modelo matemático que resulta da equação de incompressibilidade de Navier-Stokes (fluido) combinada com uma equação de transporte dominada por advecção (concentração de sedimentos)}
\note[item]{emprega um método de elementos finitos de multi-escala variacional no qual uma abordagem escalonada é utilizada para representar e simular a evolução do tempo nas equações de acoplamento entre o fluido e os sedimentos}
\end{frame}

%------------------------------------------------

\subsection*{Ambiente dos experimentos}
\begin{frame}{Ambiente dos experimentos}

\begin{columns}[t]

\column{.45\textwidth}

\vspace{-.7cm}

\begin{block}{\textit{Solver} de sedimentação}
Execução paralela ($480$~núcleos) no \textit{cluster} Lobo~Carneiro do NACAD\footnotemark{}
\end{block}

\begin{figure}
\includegraphics[width=\textwidth]{img/loboc.jpg}
\end{figure}

\setbeamercovered{invisible}

\column{.45\textwidth}

\vspace{-.7cm}

\begin{block}{Consultas}
Geração e execução em um Macbook Pro 2015:

\begin{itemize}
\item Intel Core i5 2,7~GHz
\item 4~CPUs
\item 8~GB de RAM
\end{itemize}
\end{block}

\begin{figure}
\includegraphics[width=.5\textwidth]{img/macbook.png}
\end{figure}

\end{columns}

\footnotetext{\url{https://www.nacad.ufrj.br/pt/recursos/sgiicex}}

\end{frame}

%------------------------------------------------

\subsection*{Consultas realizadas}

{\nologo
\begin{frame}{Fluxo de dados utilizado}
\begin{itemize}
\item Fluxo de dados $D^{\prime}$ = ($T^{\prime}$, $S^{\prime}$, $\phi^{\prime}$)
\end{itemize}

\begin{exampleblock}{Alguns atributos de dados de $S^{\prime}$}
\begin{table}[htb]
    \centering
    % https://tex.stackexchange.com/questions/38177/including-large-tables-in-a-beamer-frame
    \resizebox{\linewidth}{!}{% Resize table to fit within \linewidth horizontally
    \begin{tabular}{c|c|c|c}
\textbf{Conjunto de dados}                  & \textbf{Atributo de dados} & \textbf{Tipo}   & \makecell{\textbf{Exemplo} \\ \textbf{de Valor}}             \\ \hline
\multirow{3}{*}{osolversimulationtransport} & time                       & \makecell{ponto \\ flutuante} & $\dfrac{1}{10^5}$                                  \\ \cline{2-4}
                                            & t\_step                    & inteiro         & 0                                      \\ \cline{2-4}
                                            & meshwriter\_task\_id       & inteiro         & 17                                     \\ \hline
\multirow{2}{*}{omeshaggregator}            & xdmf                       & arquivo         & \texttt{$\sim$/output\_48.xmf}             \\ \cline{2-4}
                                            & n\_processors              & inteiro         & 480                                    \\ \hline
omeshrefinement                             & first\_step\_refinement    & booleano        & falso                                  \\ \hline
ovisualization                              & png                        & arquivo         & \texttt{$\sim$/image\_99.png} \\ \hline
oinputmesh                                  & mesh\_file                 & arquivo         & \texttt{$\sim$/necker3d.mesh}               \\ \hline
otimestepcontrolconfig                      & model\_name                & string          & PC11
    \end{tabular}}
    \label{tab:experiments-data-attributes}
\end{table}
\end{exampleblock}

\end{frame}
}

%------------------------------------------------

% \begin{frame}
% 
% \begin{figure}
% \includegraphics[width=.9\textwidth]{img/slides-dataflow.pdf}
% \hspace*{\fill}
% \end{figure}
% 
% \end{frame}

%------------------------------------------------

{\nologo
\begin{frame}{Consulta \#1}{Especificação}

\begin{block}{Objetivo}
Análise ao longo do \alert{tempo} da média da \alert{concentração de sedimentos} em uma linha extraída de arquivos de dados científicos.
\end{block}

\begin{block}{Argumentos para \texttt{generateSqlQuery}}
\begin{description}[\texttt{dsDestinations}]
\item[\texttt{dsOrigins}] \texttt{osolversimulationtransport}
\item[\texttt{dsDestinations}] \texttt{oline2extraction}, \texttt{omeshwriter}
\item[\texttt{type}] physical
\item[\texttt{projections}] \texttt{osolversimulationtransport.time}, \\ \texttt{AVG(oline2extraction.d)}
\item[\texttt{selections}] $\varnothing$
\end{description}
\end{block}

\end{frame}
}

%------------------------------------------------

{\nologo
\begin{frame}{Consulta \#1}{Elementos mapeados no fluxo de dados}
\begin{figure}
\includegraphics[width=\textwidth]{img/slides-dataflow-1.pdf}
\hspace*{\fill}
\end{figure}
\end{frame}
}

%------------------------------------------------

\begin{frame}[fragile]{Consulta \#1}{SQL gerado}

\begin{lstlisting}[language=sql,deletendkeywords={TIME},caption={tempo médio da geração: 40,29~ms}]
SELECT osolversimulationtransport.time, AVG(oline2extraction.d)
FROM osolversimulationtransport, oline2extraction, omeshwriter
WHERE (osolversimulationtransport.line2extraction_task_id = oline2extraction.line2extraction_task_id)
AND (osolversimulationtransport.meshwriter_task_id = omeshwriter.meshwriter_task_id)
GROUP BY osolversimulationtransport.time;
\end{lstlisting}

\end{frame}

%------------------------------------------------

\begin{frame}[fragile]{Consulta \#1}{Resultados da consulta SQL}

\begin{center}
\begin{lstlisting}[language=sqlresults,
% caption={7 tuplas, tempo médio da execução: 4,35~ms}
]
+-------------+--------------------------+
| time        | AVG(oline2extraction.d)  |
+=============+==========================+
|   1.3398483 |     1.74129306930693e-44 |
|   3.1009347 |  -1.0847045643564339e-44 |
|   5.4124618 |    7.853749801980205e-39 |
|   7.8695609 |  -1.8180726336633645e-33 |
|  10.1307669 |   1.0729463267326738e-27 |
|  12.6055519 |   1.0473414950495052e-24 |
|          15 |   -8.768540693069305e-22 |
+-------------+--------------------------+
\end{lstlisting}
\end{center}

\end{frame}

%------------------------------------------------

{\nologo
\begin{frame}{Consulta \#2}{Especificação}

\vspace{-.175cm}

\begin{block}{Objetivo}
\small{Análise da concentração de sedimentos em uma linha considerando um instante de tempo fixo e um intervalo específico de valores.}
\end{block}

\begin{block}{Argumentos para \texttt{generateSqlQuery}}
\footnotesize
\begin{description}[\texttt{dsDestinations }]
\small
\setlength\tabcolsep{1.5pt}
\item[\texttt{dsOrigins}] \texttt{osolversimulationtransport}
\item[\texttt{dsDestinations}] \texttt{oline0extraction}, \texttt{omeshwriter}
\item[\texttt{type}] physical
\item[\texttt{projections}] \makecell[l]{\texttt{osolversimulationtransporte.time}, \\ \texttt{oline0extraction.points\{0,1,2\}}, \\ \texttt{oline0extraction.d}}
\item[\texttt{selections}] \makecell[l]{\texttt{osolversimulationtransport.time < 5.5}, \\ \texttt{oline0extraction.d > 0.1}}
\end{description}
\end{block}

\end{frame}
}

%------------------------------------------------

{\nologo
\begin{frame}{Consulta \#2}{Elementos mapeados no fluxo de dados}
\begin{figure}
\includegraphics[width=\textwidth]{img/slides-dataflow-2.pdf}
\hspace*{\fill}
\end{figure}
\end{frame}
}

%------------------------------------------------

\begin{frame}[fragile]{Consulta \#2}{SQL gerado}

\vspace{-1.5cm}

\begin{lstlisting}[language=sql,deletendkeywords={TIME},caption={tempo médio da geração: 15,45~ms}]
SELECT osolversimulationtransport.time, oline0extraction.points0, oline0extraction.points1, oline0extraction.points2, oline0extraction.d
FROM osolversimulationtransport, oline0extraction, omeshwriter
WHERE (osolversimulationtransport.time < 5.5)
AND (oline0extraction.d > 0.1)
AND (osolversimulationtransport.line0extraction_task_id = oline0extraction.line0extraction_task_id)
AND (osolversimulationtransport.meshwriter_task_id = omeshwriter.meshwriter_task_id)
LIMIT 10;
\end{lstlisting}

\end{frame}

%------------------------------------------------

\subsection*{Observações}
\begin{frame}{Observações}

\begin{block}{QPP $\Rightarrow$ \texttt{generateSqlQuery}}
\begin{itemize}
\item[\checkmark] Resultados esperados foram obtidos
\begin{itemize}
\item[\checkmark] sintaticamente válidos
\end{itemize}
\item[\checkmark] Baixa sobrecarga de tempo para a geração das consultas $\rightarrow$ ordem de \texttt{ms}
\begin{itemize}
\item[\checkmark] pré-processamento com as estruturas de dados % os \textit{hash multimaps}
\item[\checkmark] idempotente
\end{itemize}
\item[\checkmark] intuitiva de usar
\end{itemize}
\end{block}

\end{frame}

%------------------------------------------------

\section{Conclusão}

\subsection*{Trabalhos futuros}
\begin{frame}{Trabalhos futuros}
\begin{itemize}
  \item interface gráfica (visualizador)
  \vfill
  \item validação das especificações do usuário
  \vfill
  \item permitir subconsultas
  \vfill
  \item suportar mais cláusulas SQL:
    \begin{itemize}
      \item agregados, \textit{e.g.} \texttt{COUNT(*)} e \texttt{GROUP BY}
      \item \texttt{SELECT DISTINCT}
      \item \texttt{LIMIT}
    \end{itemize}
  \vfill
  \item publicar um \emph{paper}~(!)
\end{itemize}

\note[item]{visualizador: Débora, em andamento. Boa usabilidade. Mouse.}
\note[item]{validação: garantir que conjuntos de dados, atributos de dados, e etc. existam; detectar erros; sintática e semanticamente.}
\note[item]{nome da conferência para o \emph{paper}: VLDB (VERY LARGE DATA BASES) 2018 \url{http://vldb2018.lncc.br/call-for-demostrations.html}}
\end{frame}

%------------------------------------------------

\subsection*{Considerações finais}

{\nologo
\begin{frame}[t,squeeze]{Considerações finais}

\footnotesize

% Vitor: O importante é enfatizar o rastreio do fluxo de elementos de dados em uma simulação computacional por meio da abstração de fluxo de dados (no nível lógico). Do ponto de vista de implementação, temos que valorizar as junções "automáticas" (a partir da análise do mapeamento de atributos) e a facilidade em desenvolver uma consulta na nossa abordagem.

% \onslide<1>{
\begin{block}{Simulação computacional...}
\begin{itemize}
\item Fluxo, conjuntos, transformações, atributos de dados\ldots{}
\item Análise de dados de domínio para simulações subsequentes
\item Dados de proveniência (prospectiva, retrospectiva), \textit{workflows}\ldots{}
\end{itemize}
\end{block}
% }

\vfill

% \onslide<2>{
\begin{alertblock}{ARMFUL $\Rightarrow$ DfAnalyzer $\Rightarrow$ QPP}
\begin{itemize}
\item \alert{Rastreio} do fluxo de elementos de dados
\item \alert{Abstração} do fluxo de dados (níveis físico, lógico e híbrido)
\item Mapeamento de atributos $\rightarrow$ \alert{junções ``automáticas''}
\item \alert{Facilidade} em desenvolver consultas
\item Não há necessidade de se conhecer o esquema do BD
\item Baixa sobrecarga de execução
\end{itemize}
\end{alertblock}
% }

\note[item]{arquivos científicos}
\note[item]{recapitular o conteúdo conceitual abordado}
\note[item]{enfatizar as vantagens do QPP}
\note[item]{defender o propósito do trabalho}
\note[item]{menor curva de aprendizado, maior usabilidade}
\end{frame}
}

%------------------------------------------------

\begin{frame}
\Huge{\centerline{Obrigado!}}

\note[item]{agradecer a atenção das pessoas}
\end{frame}

%------------------------------------------------

% https://tex.stackexchange.com/questions/30461/beamer-nonumber-equivalent-for-slides
\appendix

% Blank slide

\begin{frame}[plain]
\end{frame}

%------------------------------------------------

{
\setbeamertemplate{footline}{}
\begin{withoutheadline}
\begin{frame}{Referências}
\footnotesize{
\begin{thebibliography}{99} % Beamer does not support BibTeX so references must be inserted manually as below

\bibitem[Silva, 2017]{p1} Vítor Silva, Marta Mattoso, Daniel Oliveira et al (2017)
\newblock Raw data queries during data-intensive parallel workflow execution
\newblock \emph{Future Generation Computer Systems}

\bibitem[Silva, 2016]{p1} Vítor Silva, Marta Mattoso, Daniel Oliveira et al (2016)
\newblock In Situ Data Steering on Sedimentation Simulation with Provenance Data
\newblock \emph{SC: High Performance Computing, Networking, Storage and Analysis}

\bibitem[Silva, 2015]{p1} Vítor Silva, Marta Mattoso, Daniel Oliveira et al (2015)
\newblock Analyzing related raw data files through dataflows
\newblock \emph{Concurrency and Computation: Practice and Experience}

\end{thebibliography}
}
\end{frame}
\end{withoutheadline}
}

%------------------------------------------------

% {
% \setbeamertemplate{footline}{}
% \begin{frame}{Slides}
% Disponíveis em: \url{https://speakerdeck.com/thiagowfx}
% \end{frame}
% }

%------------------------------------------------

\end{document}

%----------------------------------------------------------------------------------------
%	PRESENTATION SLIDES
%----------------------------------------------------------------------------------------

%------------------------------------------------
% \section{Section Example} Sections can be created in order to organize your presentation into discrete blocks, all sections and subsections are automatically printed in the table of contents as an overview of the talk

% \subsection{Subsection Example} % A subsection can be created just before a set of slides with a common theme to further break down your presentation into chunks
%------------------------------------------------

%---------------------------
% SLIDES BASICS:
%---------------------------

%------------------------------------------
% \tableofcontents[currentsection,currentsubsection]
%
% \begin[fragile,label=frame1]{frame}{Title}{Subtitle}
%
% \frametitle{Title}
% \framesubtitle{Subtitle}
%
% \begin{itemize}[<+->]
% \item Item 1.
% \item Item 2.
% \begin{itemize}[<.->]
%   \item<.> Sub-Item 1.
%   \item Sub-Item 2.
% \end{itemize}
%
% \end{itemize}
%
% \begin{enumerate}
% \item Item 1.
% \item Item 2.
% \end{enumerate}
%
% \begin{block}{Block 1}
% Lorem ipsum dolor.
% \end{block}
%
% \begin{alertblock}{Block 1}
% Lorem ipsum dolor.
% \end{alertblock}
%
% \begin{exampleblock}{Block 1}
% Lorem ipsum dolor.
% \end{exampleblock}
%
% \note[item]{Note that this slide is boring.}
% 
% \end{frame}
%------------------------------------------
%
%
% Antecipar:
% - dificuldades técnicas
% - como isso efetivamente auxilia o usuário
% - processador de consultas (otimização)